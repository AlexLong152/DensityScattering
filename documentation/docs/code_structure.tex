\documentclass[11pt]{article}
\usepackage[margin=1in]{geometry}
\usepackage{amsmath}
\usepackage{hyperref}
\usepackage{enumitem}

\title{Code Structure: Twobody Nuclear Reaction Calculations}
\author{}
\date{\today}

\begin{document}

\maketitle

\section{Overview}

The twobody code computes nuclear reaction amplitudes by convoluting process-specific kernels with 2N density matrices. The code is split into two independent components that communicate through well-defined interfaces.

\subsection{Two-Component Architecture}

\begin{description}
    \item[Mantle Code] (\texttt{varsub-twobodyvia2Ndensity/}): Process-independent framework that handles quantum number summations, angular integrations, and density matrix interpolation. Works in units of \textbf{fm}.

    \item[Kernel Code]: Process-specific physics implementation. Works in units of \textbf{MeV}. Two implementations are currently available:
    \begin{itemize}
        \item \texttt{varsub-PionPhotoProdThresh.twobody/}: Pion photoproduction at threshold
        \item \texttt{varsub-PionPion.twobody/}: Pion-pion scattering
    \end{itemize}
\end{description}

\subsection{Unit Conversion}

The mantle code (fm) and kernel code (MeV) use different unit systems. Unit conversion occurs in \texttt{finalstatesums.twobodyvia2Ndensity.f} via the factor $(HC)^3/(2\pi)^3$. If the kernel has units MeV$^{-n}$, the final output has units MeV$^{3-n}$.

\section{Call Hierarchy}

\subsection{Main Control Flow}

\begin{enumerate}[leftmargin=2cm]
    \item[\textbf{Level 1:}] \texttt{varsub-main.twobodyvia2Ndensity.f} \\
    Main program: controls energy/angle loops, reads input/density files, outputs results

    \item[\textbf{Level 2:}] \texttt{varsub-finalstatesums.twobodyvia2Ndensity.f} \\
    Function: \texttt{twobodyfinalstatesumsvia2Ndensity()} \\
    Loops over final state quantum numbers, accumulates contributions

    \item[\textbf{Level 3:}] \texttt{varsub-calculate2BI2.f} \\
    Function: \texttt{Calculate2BIntegralI2()} \\
    Performs angular integrals and spin sums, interpolates density matrix

    \item[\textbf{Level 4:}] \texttt{varsub-2Bkernel.PionPhotoProdThresh.f} \\
    Function: \texttt{Calc2Bspinisospintrans()} \\
    Computes process-specific kernel amplitudes

    \item[\textbf{Level 5:}] \texttt{varsub-2Bspinsym.PionPhotoProdThresh.f} (and \texttt{*asym.f}) \\
    Functions: \texttt{CalcKernel2B*()} and \texttt{StaticKernel*()} \\
    Evaluate spin structures for specific diagrams
\end{enumerate}

\section{File Descriptions}

\subsection{Mantle Code Files}

\subsubsection{\texttt{varsub-main.twobodyvia2Ndensity.f}}

\textbf{Purpose:} Program entry point and orchestration

\textbf{Responsibilities:}
\begin{itemize}
    \item Parse input file (energies, angles, nucleus properties, quadrature parameters)
    \item Set up radial and angular quadrature grids
    \item Loop over energies and scattering angles
    \item Read 2N density matrix from HDF5 files
    \item Loop over initial quantum numbers (mt12, j12, s12, l12, m12, ip12)
    \item Call \texttt{twobodyfinalstatesumsvia2Ndensity()} for each configuration
    \item Write results to output file
\end{itemize}

\textbf{Key Variables:}
\begin{itemize}
    \item \texttt{Result(extQnum, twoMzp, twoMz)}: Output array storing computed amplitudes
    \item \texttt{P12MAG(ip12)}: Radial momentum grid points
    \item \texttt{AP12MAG(ip12)}: Radial momentum integration weights
    \item \texttt{rhoDensity}: 2N density matrix (read from file, stored in module)
\end{itemize}

\subsubsection{\texttt{varsub-finalstatesums.twobodyvia2Ndensity.f}}

\textbf{Purpose:} Sum over final state quantum numbers

\textbf{Responsibilities:}
\begin{itemize}
    \item Loop over final state quantum numbers: j12p, s12p, l12p, m12p, ip12p, twoMzp, twoMz
    \item Call \texttt{Calculate2BIntegralI2()} to compute angular integrals
    \item Multiply by integration weights (momentum weights, phase space factors)
    \item Apply unit conversion factor: $(HC)^3/(2\pi)^3$
    \item Accumulate into \texttt{Result()} array
\end{itemize}

\textbf{Key Constraint:} mt12p = mt12 (isospin projection is conserved)

\subsubsection{\texttt{varsub-calculate2BI2.f}}

\textbf{Purpose:} Perform angular integrals and spin sums

\textbf{Responsibilities:}
\begin{itemize}
    \item Loop over spin projections (ms, msp) of the (12) subsystem
    \item Nested loops over angular quadrature points: ($\theta_{12}$, $\phi_{12}$) and ($\theta_{12}'$, $\phi_{12}'$)
    \item Convert spherical to Cartesian coordinates
    \item Compute spherical harmonics
    \item Call \texttt{Calc2Bspinisospintrans()} to get kernel amplitudes
    \item Interpolate density matrix at required momentum values (bilinear interpolation)
    \item Weight by spherical harmonics, quadrature weights, and Clebsch-Gordan coefficients
    \item Sum over all angular points and spin projections
\end{itemize}

\textbf{Quadrature Options:}
\begin{itemize}
    \item Gaussian: Separate 1D quadratures for $\theta$ and $\phi$
    \item Lebedev-Laikov: Spherical quadrature on unit sphere
\end{itemize}

\subsubsection{Supporting Files in \texttt{varsub-twobodyvia2Ndensity/}}

\begin{itemize}
    \item \texttt{varsub-read2Ndensity.f}: Reads HDF5 density files into module variables
    \item \texttt{varsub-setquads.f}: Sets up angular quadrature grids
    \item \texttt{varsub-LebedevLaikov.f}: Lebedev-Laikov quadrature implementation
    \item \texttt{varsub-spinstructures.f}: General spin algebra routines (\texttt{singlesigmasym()}, \texttt{doublesigmasym()})
\end{itemize}

\subsection{Kernel Code Files}

\subsubsection{\texttt{varsub-2Bkernel.PionPhotoProdThresh.f}}

\textbf{Purpose:} Main kernel computation for pion photoproduction

\textbf{Responsibilities:}
\begin{itemize}
    \item \texttt{KernelGreeting()}: Print process name and version to stdout
    \item \texttt{KernelFarewell()}: Print description of computed quantities and units
    \item \texttt{Calc2Bspinisospintrans()}: Main kernel calculation routine
    \begin{itemize}
        \item Receives momentum vectors (pVec, uVec) and quantum numbers
        \item Performs variable substitution to simplify momentum transfers
        \item Computes momentum vectors for use in diagrams (qVec, qpVec, kVec, kpVec)
        \item Calls diagram-specific routines based on \texttt{calctype} parameter
        \item Returns \texttt{Kernel2B(diagNum, extQnum, s12p, msp, s12, ms)}
    \end{itemize}
    \item \texttt{getDiagAB()}: Computes leading order contributions
    \item \texttt{getStaticDiags()}: Computes next-to-leading order corrections
\end{itemize}

\textbf{Chiral Order Organization:}
\begin{itemize}
    \item O($\delta^2$): Leading order contributions
    \item O($\delta^4$): Next-to-leading order corrections
\end{itemize}

Calculation terminates at the order specified by \texttt{calctype}.

\textbf{Variable Substitution:}

The code uses $\vec{u} = \vec{p}_{12} - \vec{p}_{12}' + \vec{k}/2$ as the integration variable instead of $\vec{p}_{12}'$. This simplifies the momentum transfer and introduces a Jacobian factor of $-1$.

\subsubsection{\texttt{varsub-2Bspinsym.PionPhotoProdThresh.f}}

\textbf{Purpose:} Spin structures for symmetric diagrams (s12p = s12)

\textbf{Functions:}
\begin{itemize}
    \item \texttt{CalcKernel2BAsym()}: Leading order contribution (type A)
    \item \texttt{CalcKernel2BBsymVec()}: Leading order contribution (type B)
    \item \texttt{StaticKernelAsym()}: NLO correction (type A)
    \item \texttt{StaticKernelBsym()}: NLO correction (type B)
    \item \texttt{StaticKernelCsym()}: NLO correction (type C)
    \item \texttt{StaticKernelDsym()}: NLO correction (type D)
    \item \texttt{StaticKernelEsym()}: NLO correction (type E)
\end{itemize}

Each function receives prefactors and momentum vectors, computes spin matrix elements, and adds contributions to the kernel array.

\subsubsection{\texttt{varsub-2Bspinasy.PionPhotoProdThresh.f}}

\textbf{Purpose:} Spin structures for antisymmetric diagrams (s12p $\neq$ s12)

Contains analogous functions to the symmetric file but for spin-flip transitions:
\texttt{CalcKernel2BAasy()}, \texttt{CalcKernel2BBasyVec()}, \texttt{StaticKernelAasym()}, etc.

\subsubsection{Other Kernel Files}

\begin{itemize}
    \item \texttt{varsub-calculateQs.PionPhotoProdThresh.f}: Momentum kinematics calculations
    \item \texttt{varsub-usesymmetries.PionPhotoProdThresh.f}: Symmetry relations (currently not used)
    \item \texttt{readinput.twobody.PionPhotoProdThresh.f}: Kernel-specific input parsing
\end{itemize}

\subsection{Pion-Pion Scattering Kernel Files}

The pion-pion scattering kernel (\texttt{varsub-PionPion.twobody/}) implements elastic pion scattering from nuclei based on chiral perturbation theory. The structure closely parallels the photoproduction kernel but with different physics and kinematics.

\subsubsection{\texttt{varsub-2Bkernel.PionPion.f}}

\textbf{Purpose:} Main kernel computation for pion-pion scattering

\textbf{Responsibilities:}
\begin{itemize}
    \item \texttt{KernelGreeting()}: Print process name, compute probe energy from gamma energy
    \begin{itemize}
        \item Converts lab-frame energy to probe pion energy via relativistic kinematics
        \item Handles near-threshold kinematics where $k^2$ may be slightly negative
    \end{itemize}
    \item \texttt{KernelFarewell()}: Print description (references BKM review equation 5.30)
    \item \texttt{Calc2Bspinisospintrans()}: Main kernel calculation routine
    \begin{itemize}
        \item Performs variable substitution: $\vec{u} = \vec{p}_{12} - \vec{p}_{12}' + (\vec{k} + \vec{k}')/2$
        \item Calculates final pion momentum $\vec{k}'$ using \texttt{calculateqs2Mass()}
        \item Computes momentum transfer vector $\vec{q}$
        \item Applies overall prefactor: $8\pi\sqrt{s}$ where $\sqrt{s} = E_{\text{nuc}} + E_{\pi}$
        \item Calls \texttt{getDiagAB()} to compute kernel contributions
        \item Returns kernel in units MeV$^{-3}$ (so Result has units MeV$^0$)
    \end{itemize}
    \item \texttt{getDiagAB()}: Computes diagram contributions (A, B, C)
    \begin{itemize}
        \item Diagram A: Contact interaction, prefactor $\propto 1/q^2$
        \item Diagram B: Single-sigma structure, prefactor $\propto 1/(q^2 + m_\pi^2)$
        \item Diagram C: Double-sigma structure, prefactor $\propto 1/(q^2 + m_\pi^2)^2$
        \item All diagrams include isospin factors depending on $(t_{12}, m_{t12})$
        \item Separates symmetric (s12p = s12) and antisymmetric (s12p $\neq$ s12) contributions
    \end{itemize}
\end{itemize}

\textbf{Physics Basis:}
\begin{itemize}
    \item Based on BKM (Bernard-Kaiser-Mei{\ss}ner) review equation 5.30
    \item Implements chiral EFT for pion-nucleon interactions
    \item Reduced mass: $\mu = m_\pi/m_N$
    \item Base prefactor: $\frac{1}{32(1+\mu)(\pi f_\pi)^4}$ where $f_\pi = 92.42$ MeV
\end{itemize}

\textbf{Kinematics:}
\begin{itemize}
    \item Elastic scattering: $\pi + N \to \pi + N$
    \item Uses center-of-mass kinematics via \texttt{calculateqs2Mass()}
    \item Momentum vectors: $\vec{p}, \vec{p}', \vec{k}, \vec{k}'$ for initial/final nucleon and pion momenta
    \item Momentum transfer: $\vec{q} = \vec{p} - \vec{p}' + (\vec{k} + \vec{k}')/2$
\end{itemize}

\subsubsection{\texttt{varsub-2Bspinsym.PionPion.f}}

\textbf{Purpose:} Spin structures for symmetric diagrams (s12p = s12)

\textbf{Functions:}
\begin{itemize}
    \item \texttt{CalcKernel2BAsym()}: Diagram A contribution
    \begin{itemize}
        \item Pure isospin structure, no momentum dependence
        \item Includes factor $(-1)^{t_{12}} \delta_{m_{t12},0}$
        \item Diagonal in spin: $\delta_{s_{12}',s_{12}} \delta_{m_{s}',m_s}$
    \end{itemize}
    \item \texttt{CalcKernel2BBsym()}: Diagram B contribution
    \begin{itemize}
        \item Calls \texttt{doublesigmasym()} for $\vec{\sigma}_1 \cdot \vec{q} \; \vec{\sigma}_2 \cdot \vec{q}$ structure
        \item Isospin factor: $(2t_{12}(t_{12}+1) - 3)$
    \end{itemize}
    \item \texttt{CalcKernel2BCsym()}: Diagram C contribution
    \begin{itemize}
        \item Same spin structure as B but different momentum dependence
        \item Includes $(q^2 + m_\pi^2)^{-2}$ instead of $(q^2 + m_\pi^2)^{-1}$
    \end{itemize}
    \item \texttt{CalcKernel2BDsym()}: Diagram D (not currently used)
\end{itemize}

\subsubsection{\texttt{varsub-2Bspinasy.PionPion.f}}

\textbf{Purpose:} Spin structures for antisymmetric diagrams (s12p $\neq$ s12)

Contains analogous functions for spin-flip transitions:
\begin{itemize}
    \item \texttt{CalcKernel2BAasy()}: Spin-flip version of diagram A
    \item \texttt{CalcKernel2BBasy()}: Uses \texttt{doublesigmaasy()} instead of \texttt{doublesigmasym()}
    \item \texttt{CalcKernel2BCasy()}: Spin-flip version of diagram C
    \item \texttt{CalcKernel2BDasy()}: Diagram D (not implemented)
\end{itemize}

\subsubsection{\texttt{varsub-calculateQs.PionPion.f}}

\textbf{Purpose:} Momentum kinematics for pion-pion scattering

\textbf{Key Subroutines:}
\begin{itemize}
    \item \texttt{calculateqs2Mass()}: Elastic scattering kinematics
    \begin{itemize}
        \item Input: momenta $\vec{p}, \vec{p}', \vec{k}$ and masses $m_1, m_2, m_3, m_4$
        \item Calculates final pion momentum $\vec{k}'$ from energy-momentum conservation
        \item Uses Mandelstam $s = (E_1 + E_2)^2$ in CM frame
        \item Output momentum magnitude: $|\vec{k}'| = \sqrt{E_4^2 - m_4^2}$
        \item Direction: $\vec{k}' = (0, |\vec{k}'|\sin\theta_{\text{cm}}, |\vec{k}'|\cos\theta_{\text{cm}})$
    \end{itemize}
    \item \texttt{CalculateQs()}: Legacy routine (not currently used in PionPion)
\end{itemize}

\textbf{Kinematic Relations:}
\begin{itemize}
    \item $E_1 = \sqrt{m_1^2 + \vec{k}^2}$, $E_2 = \sqrt{m_2^2 + \vec{k}^2}$
    \item $\sqrt{s} = E_1 + E_2$
    \item $E_3 = \frac{1}{2\sqrt{s}}(s + m_3^2 - m_4^2)$
    \item $E_4 = \sqrt{s} - E_3$
\end{itemize}

\subsubsection{Other PionPion Kernel Files}

\begin{itemize}
    \item \texttt{varsub-usesymmetries.PionPion.f}: Symmetry relations (currently not used)
    \item \texttt{varsub-readinput.twobody.PionPion.f}: Kernel-specific input parsing
\end{itemize}

\subsection{Comparison: Photoproduction vs Pion-Pion}

\begin{center}
\begin{tabular}{|l|l|l|}
\hline
\textbf{Feature} & \textbf{Photoproduction} & \textbf{Pion-Pion} \\
\hline
Process & $\gamma + N \to \pi + N$ & $\pi + N \to \pi + N$ \\
\hline
Theory & Threshold expansion & Chiral EFT (BKM 5.30) \\
\hline
Orders & O($\delta^2$), O($\delta^4$) & Single order \\
\hline
Diagrams & A, B, Static (A--E) & A, B, C \\
\hline
Kinematics & Photoproduction threshold & Elastic scattering \\
\hline
Output units & fm$^{-1}$ & MeV$^0$ (dimensionless) \\
\hline
Kernel units & MeV$^{-2}$ & MeV$^{-3}$ \\
\hline
\end{tabular}
\end{center}

\textbf{Common Features:}
\begin{itemize}
    \item Both use variable substitution in momentum integration
    \item Both separate symmetric/antisymmetric spin contributions
    \item Both use same mantle code infrastructure
    \item Both implement isospin algebra
    \item Both call \texttt{doublesigmasym()/doublesigmaasy()} from \texttt{varsub-spinstructures.f}
\end{itemize}

\section{Data Flow}

\subsection{Input Data}

\begin{enumerate}
    \item \textbf{Input file} (text): Energies, angles, quadrature parameters, file paths
    \item \textbf{Density file} (HDF5): 2N density matrix $\rho(p_{12}, p_{12}', \text{quantum numbers})$
\end{enumerate}

\subsection{Computational Flow}

\begin{verbatim}
Main Program
    | (loops: energy, angle, quantum numbers)
    |
    +---> Read Density Matrix (once per energy/angle)
    |
    +---> twobodyfinalstatesumsvia2Ndensity()
            | (loop: final state quantum numbers)
            |
            +---> Calculate2BIntegralI2()
                    | (loops: spin projections, angles)
                    | (interpolate density)
                    |
                    +---> Calc2Bspinisospintrans()
                            | (compute momenta)
                            |
                            +---> getDiagAB()
                            |       +---> CalcKernel2B*sym/asy()
                            |
                            +---> getStaticDiags()
                                    +---> StaticKernel*sym/asy()
                                    +---> (multiple contributions)
                    |
                    | <--- returns: Kernel2B(...)
                    |
                | (weight and sum)
            |
            | <--- returns: Int2B(diagNum, extQnum)
            |
        | (accumulate with weights)
    |
    | <--- updates: Result(extQnum, twoMzp, twoMz)
    |
+---> Write Results to File
\end{verbatim}

\subsection{Output Data}

\begin{enumerate}
    \item \textbf{Output file} (text): \texttt{Result(extQnum, twoMzp, twoMz)} for each energy and angle
    \item \textbf{extQnum}: Index for external quantum numbers
    \begin{itemize}
        \item For photoproduction: polarization indices (1=x, 2=y, 3=z)
        \item For pion-pion: typically 1 (no polarization dependence)
    \end{itemize}
    \item \textbf{twoMzp, twoMz}: Initial and final nuclear spin projections (times 2)
\end{enumerate}

\textbf{Output units vary by process:}
\begin{itemize}
    \item Pion photoproduction: fm$^{-1}$ (proportional to $F_{TL}$ functions)
    \item Pion-pion scattering: MeV$^0$ (dimensionless, proportional to scattering amplitude)
\end{itemize}

\section{Key Interfaces}

\subsection{Mantle $\rightarrow$ Kernel Interface}

The mantle code calls \texttt{Calc2Bspinisospintrans()} with:

\textbf{Inputs:}
\begin{itemize}
    \item \texttt{pVec(3)}: Physical momentum vector in MeV
    \item \texttt{uVec(3)}: Integration variable vector in MeV
    \item Quantum numbers: \texttt{ml12, ml12p, t12, mt12, t12p, mt12p, l12, s12, l12p, s12p}
    \item Kinematics: \texttt{thetacm, Eprobe, Mnucl}
    \item Control: \texttt{calctype, numDiagrams, extQnumlimit}
\end{itemize}

\textbf{Outputs:}
\begin{itemize}
    \item \texttt{Kernel2B(diagNum, extQnum, s12p, msp, s12, ms)}: Complex kernel amplitudes
    \item \texttt{ppVecs(diagNum, 1:3)}: Transformed momentum vectors for each diagram
\end{itemize}

The kernel has no knowledge of:
\begin{itemize}
    \item Density matrices
    \item Angular integrals or quadratures
    \item Final state summations
    \item Output file formats
\end{itemize}

\subsection{Density Matrix Interface}

The density matrix is read once per energy/angle and stored in a Fortran module (\texttt{CompDens}). It is accessed throughout the calculation via:

\begin{itemize}
    \item \texttt{rhoDensity(ip12, ip12p, rindx)}: 3D array
    \item \texttt{ip12, ip12p}: Momentum grid indices
    \item \texttt{rindx}: Composite index for all quantum numbers of the (12) channel
\end{itemize}

The mantle code performs bilinear interpolation to evaluate the density at arbitrary momentum values needed for the kernel.

\section{Extensibility}

\subsection{Adding a New Process}

To compute a different reaction (e.g., Compton scattering, kaon production), follow these steps. The pion photoproduction and pion-pion scattering kernels serve as reference implementations.

\begin{enumerate}
    \item Create new kernel directory: \texttt{varsub-<Process>.twobody/}
    \item Implement required subroutines:
    \begin{itemize}
        \item \texttt{KernelGreeting()}: Identify your process, compute kinematic variables
        \item \texttt{KernelFarewell()}: Describe your output and units
        \item \texttt{Calc2Bspinisospintrans()}: Main kernel computation
        \begin{itemize}
            \item Set up kinematics and momentum vectors
            \item Call diagram calculation routines
            \item Apply overall prefactors
            \item Return kernel with proper units
        \end{itemize}
        \item Diagram-specific spin structure routines (sym/asy versions)
        \item Kinematic calculation routines (analogous to \texttt{calculateqs2Mass()})
    \end{itemize}
    \item Ensure proper unit handling:
    \begin{itemize}
        \item Kernel should work in MeV
        \item Choose units so Result() has desired output units
        \item Remember: Result units = MeV$^{3-n}$ if kernel has units MeV$^{-n}$
    \end{itemize}
    \item The mantle code remains completely unchanged
    \item Update Makefile to link new kernel files
\end{enumerate}

\textbf{Key Design Principles:}
\begin{itemize}
    \item Keep kernel code independent of mantle infrastructure
    \item Separate symmetric and antisymmetric spin contributions
    \item Use variable substitution if it simplifies integrals (include Jacobian!)
    \item Implement isospin algebra explicitly in diagram prefactors
    \item Document physics basis (review paper, equation numbers)
\end{itemize}

\subsection{Adding Higher Orders}

To add higher order contributions (e.g., O($\delta^6$)):

\begin{enumerate}
    \item Add new diagram calculation routines in kernel files
    \item Update \texttt{Calc2Bspinisospintrans()} to call new diagrams when appropriate \texttt{calctype} is set
    \item Update \texttt{calctype.def} to include new order definitions
    \item No changes needed to mantle code structure
\end{enumerate}

\section{Important Conventions}

\subsection{Quantum Numbers}

\begin{itemize}
    \item Nuclear spin quantum numbers are stored as \texttt{twoMz} = $2M_z$ (integers)
    \item Allows half-integer spins to be represented exactly
    \item (12) subsystem quantum numbers (j12, s12, l12, etc.) are integers (0 or 1 for spin)
    \item All loops over \texttt{twoMz} use steps of 2: \texttt{do twoMz = twoSnucl, -twoSnucl, -2}
\end{itemize}

\subsection{Array Indexing}

\begin{itemize}
    \item \texttt{Result(extQnum, twoMzp, twoMz)}
    \begin{itemize}
        \item extQnum: 1 to extQnumlimit (typically 1--3 for polarizations)
        \item twoMzp, twoMz: -twoSnucl to +twoSnucl in steps of 2
    \end{itemize}
    \item \texttt{Kernel2B(diagNum, extQnum, s12p, msp, s12, ms)}
    \begin{itemize}
        \item diagNum: 1 to numDiagrams (currently 1 for this process)
        \item s12p, s12: 0 or 1
        \item msp, ms: -1, 0, or +1 (but only -s12 to +s12 are physical)
    \end{itemize}
\end{itemize}

\subsection{Module Usage}

The code uses a Fortran module \texttt{CompDens} (defined in \texttt{CompDens.F90}) to share density matrix data without passing large arrays through function arguments. This module is imported via \texttt{USE CompDens} in relevant files.

\section{Summary}

The code architecture cleanly separates process-independent framework (mantle) from process-specific physics (kernel):

\begin{description}
    \item[Mantle] Handles all bookkeeping: quantum number loops, angular integrals, density interpolation, unit conversion, I/O. Works in fm.

    \item[Kernel] Implements physics: Feynman diagrams, spin structures, momentum kinematics. Works in MeV. Two kernels currently implemented:
    \begin{itemize}
        \item \textbf{Pion Photoproduction}: Threshold expansion with O($\delta^2$) and O($\delta^4$) contributions
        \item \textbf{Pion-Pion Scattering}: Chiral EFT elastic scattering (BKM review)
    \end{itemize}

    \item[Interface] Well-defined: kernel receives momenta and quantum numbers, returns amplitudes. No shared knowledge of implementation details.

    \item[Result] Reusable framework. Same mantle code used for multiple processes by swapping kernel implementation. This document demonstrates extensibility with two working examples.
\end{description}

\textbf{Code Organization:}
\begin{itemize}
    \item Mantle files: \texttt{varsub-twobodyvia2Ndensity/*}
    \item Photoproduction kernel: \texttt{varsub-PionPhotoProdThresh.twobody/*}
    \item Pion-pion kernel: \texttt{varsub-PionPion.twobody/*}
    \item Common utilities: \texttt{common-densities/*}
\end{itemize}

\end{document}
